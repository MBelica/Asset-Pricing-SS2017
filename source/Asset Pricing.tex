\documentclass[12pt]{extreport} % Schriftgröße: 8pt, 9pt, 10pt, 11pt, 12pt, 14pt, 17pt oder 20pt

%% Packages
\usepackage{scrextend}
\usepackage{amssymb}
\usepackage{amsthm}
\usepackage{booktabs}
\usepackage{chngcntr}
\usepackage{cmap}
\usepackage{color}
\usepackage{csquotes}
\usepackage{enumitem}
\usepackage{float}
\usepackage{hyperref}
\usepackage{ulem}
\usepackage{lmodern}
\usepackage{makeidx}
\usepackage{amsmath}
\usepackage{mathtools}
\usepackage{xpatch}
\usepackage{pgfplots}
\pgfplotsset{compat=1.7}
\usetikzlibrary{calc}	
\usetikzlibrary{matrix}	

% Language Setup (English)
\usepackage[utf8]{inputenc} 
\usepackage[T1]{fontenc} 
\usepackage[english]{babel}

% Options
\makeatletter%%  
  % Linkfarbe, {0,0.35,0.35} für Türkis, {0,0,0} für Schwarz, {1,0,0} für Rot, {0,0,0.85} für Blau
  \definecolor{linkcolor}{rgb}{0,0.35,0.35}
  % Zeilenabstand für bessere Leserlichkeit
  \def\mystretch{1.2} 
  % Publisher definieren
  \newcommand\publishers[1]{\newcommand\@publishers{#1}} 
  % Enumerate im 1. Level: \alph für a), b), ...
  \renewcommand{\labelenumi}{\alph{enumi})} 
  % Enumerate im 2. Level: \roman für (i), (ii), ...
  \renewcommand{\labelenumii}{(\roman{enumii})}
  % Zeileneinrückung am Anfang des Absatzes
  \setlength{\parindent}{0pt} 
  % Für das Proof-Environment: 'Beweis:' anstatt 'Beweis.'
  \xpatchcmd{\proof}{\@addpunct{.}}{\@addpunct{:}}{}{} 
  % Nummerierung der Bilder, z.B.: Abbildung 4.1
  \@ifundefined{thechapter}{}{\def\thefigure{\thechapter.\arabic{figure}}} 
  % Chapter-Nummerierung beginnen bei (0):
  \setcounter{chapter}{0}
  % Chapter-Nummerierung
  \renewcommand\thechapter{\Roman{chapter}}
\makeatother%

% Meta Setup 
\title{Asset Pricing}
\author{Prof. Marliese Uhrig-Homburg}
\date{Sommersemester 2017}
\publishers{Karlsruher Institut für Technologie}

%% Math. Definitiones
\newcommand{\C}{\mathbb{C}}
\newcommand{\N}{\mathbb{N}}
\newcommand{\Q}{\mathbb{Q}}
\newcommand{\R}{\mathbb{R}}
\newcommand{\Z}{\mathbb{Z}}
\newcommand{\DO}[1]{\mathcal{D}\left( {#1} \right)}
\newcommand{\RO}[1]{\mathcal{R}\left( {#1} \right)}

\newtheoremstyle{named}{}{}{\normalfont}{}{\bfseries}{:}{0.25em}{#2 \thmnote{#3}}
\newtheoremstyle{nnamed}{}{}{\normalfont}{}{\bfseries}{:}{0.25em}{\thmnote{#3}}
\newtheoremstyle{itshape}{}{}{\itshape}{}{\bfseries}{:}{ }{}
\newtheoremstyle{normal}{}{}{\normalfont}{}{\bfseries}{:}{ }{}
\renewcommand*{\qed}{\hfill\ensuremath{\square}}

\theoremstyle{named}
\newtheorem{unnamedtheorem}{Theorem} \counterwithin{unnamedtheorem}{chapter}
\theoremstyle{nnamed}
\newtheorem*{unnamedtheorem*}{Theorem} 

\theoremstyle{itshape}
\newtheorem{definition}[unnamedtheorem]{Definition}

\theoremstyle{normal}
\newtheorem*{recall}{Recall}
\newtheorem*{example}{Example}
\newtheorem*{remark}{Remark}
\newtheorem*{satz}{Satz}
\newtheorem*{bemerkung}{Bemerkung}
\newtheorem*{beispiel}{Beispiel}


%% Template
\makeatletter%
\DeclareUnicodeCharacter{00A0}{ } \pgfplotsset{compat=1.7} \hypersetup{colorlinks,breaklinks, urlcolor=linkcolor, linkcolor=linkcolor, pdftitle=\@title, pdfauthor=\@author, pdfsubject=\@title, pdfcreator=\@publishers}\DeclareOption*{\PassOptionsToClass{\CurrentOption}{report}} \ProcessOptions \def\baselinestretch{\mystretch} \setlength{\oddsidemargin}{0.125in} \setlength{\evensidemargin}{0.125in} \setlength{\topmargin}{0.5in} \setlength{\textwidth}{6.25in} \setlength{\textheight}{8in} \addtolength{\topmargin}{-\headheight} \addtolength{\topmargin}{-\headsep} \def\pulldownheader{ \addtolength{\topmargin}{\headheight} \addtolength{\topmargin}{\headsep} \addtolength{\textheight}{-\headheight} \addtolength{\textheight}{-\headsep} } \def\pullupfooter{ \addtolength{\textheight}{-\footskip} } \def\ps@headings{\let\@mkboth\markboth \def\@oddfoot{} \def\@evenfoot{} \def\@oddhead{\hbox {}\sl \rightmark \hfil \rm\thepage} \def\chaptermark##1{\markright {\uppercase{\ifnum \c@secnumdepth >\m@ne \@chapapp\ \thechapter. \ \fi ##1}}} \pulldownheader } \def\ps@myheadings{\let\@mkboth\@gobbletwo \def\@oddfoot{} \def\@evenfoot{} \def\sectionmark##1{} \def\subsectionmark##1{}  \def\@evenhead{\rm \thepage\hfil\sl\leftmark\hbox {}} \def\@oddhead{\hbox{}\sl\rightmark \hfil \rm\thepage} \pulldownheader }	\def\chapter{\cleardoublepage  \thispagestyle{plain} \global\@topnum\z@ \@afterindentfalse \secdef\@chapter\@schapter} \def\@makeschapterhead#1{ {\parindent \z@ \raggedright \normalfont \interlinepenalty\@M \Huge \bfseries  #1\par\nobreak \vskip 40\p@ }} \newcommand{\indexsection}{chapter} \patchcmd{\@makechapterhead}{\vspace*{50\p@}}{}{}{}\def\Xint#1{\mathchoice
    {\XXint\displaystyle\textstyle{#1}} {\XXint\textstyle\scriptstyle{#1}} {\XXint\scriptstyle\scriptscriptstyle{#1}} {\XXint\scriptscriptstyle\scriptscriptstyle{#1}} \!\int} \def\XXint#1#2#3{{\setbox0=\hbox{$#1{#2#3}{\int}$} \vcenter{\hbox{$#2#3$}}\kern-.5\wd0}} \def\dashint{\Xint-} \def\Yint#1{\mathchoice {\YYint\displaystyle\textstyle{#1}} {\YYYint\textstyle\scriptscriptstyle{#1}} {}{} \!\int} \def\YYint#1#2#3{{\setbox0=\hbox{$#1{#2#3}{\int}$} \lower1ex\hbox{$#2#3$}\kern-.46\wd0}} \def\YYYint#1#2#3{{\setbox0=\hbox{$#1{#2#3}{\int}$}  \lower0.35ex\hbox{$#2#3$}\kern-.48\wd0}} \def\lowdashint{\Yint-} \def\Zint#1{\mathchoice {\ZZint\displaystyle\textstyle{#1}}{\ZZZint\textstyle\scriptscriptstyle{#1}} {}{} \!\int} \def\ZZint#1#2#3{{\setbox0=\hbox{$#1{#2#3}{\int}$}\raise1.15ex\hbox{$#2#3$}\kern-.57\wd0}} \def\ZZZint#1#2#3{{\setbox0=\hbox{$#1{#2#3}{\int}$} \raise0.85ex\hbox{$#2#3$}\kern-.53\wd0}} \def\highdashint{\Zint-} \DeclareRobustCommand*{\onlyattoc}[1]{} \newcommand*{\activateonlyattoc}{ \DeclareRobustCommand*{\onlyattoc}[1]{##1} } \AtBeginDocument{\addtocontents{toc} {\protect\activateonlyattoc}} \newcommand{\RightArrow}{\xRightarrow[]{ ~ ~ }} \newcommand{\LeftArrow}{\xLeftarrow[]{ ~ ~ }} \newcommand{\rightArrow}{\xrightarrow[]{ ~ ~ }} \newcommand{\leftArrow}{\xleftarrow[]{ ~ ~ }}
	% Titlepage
	\def\maketitle{ \begin{titlepage} 
			~\vspace{3cm} 
		\begin{center} {\Huge \@title} \end{center} 
	 		\vspace*{1cm} 
	 	\begin{center} {\large \@author} \end{center} 
	 	\vspace*{-0.5cm}
	 	\begin{center} \@date \end{center} 
	 		\vspace*{7cm} 
	 	\begin{center} \@publishers \end{center} 
	 		\vfill 
	\end{titlepage} }
\makeatother%

% Create Index
\makeindex 

\begin{document}

\pagenumbering{Alph}
\begin{titlepage}
	\maketitle
	\thispagestyle{empty}
\end{titlepage}

% Table of Contents
\tableofcontents
\thispagestyle{empty}

% Lecture Notes - Start 			
\pagenumbering{arabic}

\chapter{Einführung}

Es werden verschiedene Preise bzw. Renditen an Wertpapiermärkten betrachtet, unter anderem:

\begin{itemize}
	\item Aktienkurse
	\item Anleiherenditen
	\item Derivatenpreise
\end{itemize}

Warum stellen sich beobachtete Marktpreise bzw. Renditen ein? A priori bedeuten niedrige Price eine höhere Rendite. Man kann also die obige Frage auch umformulieren zu: warum einige Asset einen höheren erwarteten Return auszahlen als andere.

\subsubsection{Asset Pricing Theorie}
Es gibt zwei Aspekte die entscheidend in den Wert eines Assets einfließen. Einmal ist es die Verzögerung der Auszahlen, also wie lange man auf die Auszahlung des z.B. Wertpapiers warten muss und anderer Seits beeinflusst auch das zugrunde liegende Risiko den Wert des Assets. Auch wenn die Korrektur für die Verzögerung weniger problematisch ist, ist die Anpassung für das Risiko ein sehr viel wichtigerer Faktor für die Bewertung des Wertes eines Assets. Asset Pricing, wie viele anderer ökonomische Disziplinen, teilt den Zwiespalt in normativer zu deskriptiver Theorie. Asset Pricing liefert sowohl
\begin{itemize}
	\item Bewertungstheorien zur Erklärung der Preisbildung. Damit ist es möglich zu verstehen, warum sich Preise oder Returns so eingestellt haben, wie sie sind; als auch
	\item Schlussfolgerungen, falls Vorhersagen der Theorie $\neq$ Beobachtung. Was ein Verständnis dafür gibt, warum und wo Bewertungen sich falsch eingestellt haben und ggf. diese Schwächen ausnutzen.
\end{itemize}

%\textbf{Anpassung der Theorie erforderlich:} Positive Theorie ~\\ 
% todo dieses Positive Theorie ist da im Raum
Gerade der letzte Aspekt ist ein Grund für die Beliebtheit und vielzahligen praktischen Anwendung von Asset-Pricing. Denn eine \textbf{Fehlbewertung am Markt} ermöglicht eine \textbf{Handelsstrategie} (Normative Theorie). ~\bigskip

Diese auftrentede Fehleinschätzung von Wertpapieren stammt unter anderem aus der oftmals vorherrschenden Asymmetrie von Informationen an einem Markt. Häufig sind keine Marktpreise für Ansprüche auf zukünftige Zahlungen vorhanden
\begin{itemize}
	\item geplante Investitionsprojekte
	\item Neuemissionen von Wertpapieren
	\item Finanzinnovationen
	\item $\dotsc$
\end{itemize}

Man findet demnach die Asset Pricing Theorie oft für zwei Grundlegende Aspekte:
\begin{itemize}
	\item zur Begründung, wie hoch fairer Preis sein sollte, und
	\item als wichtige Entscheidungsgrundlage
\end{itemize}

\section{Kernprinzipien der Finanzwirtschaft}
Bewertung fußt auf den Kernprinzipien der Finanzwirtschaft:
\begin{itemize}
	\item \textbf{Primat der Zahlungen}: für eine Entscheidung sind allein Zahlung (oder Zahlungsäquivalente) relevant; dies wird manchmal auch \textit{Dominance} genannt.
	\item \textbf{Zeitwert des Geldes}: der Wert einer Zahlung hängt davon ab, wann sie erfolgt
	\item \textbf{Ertrag vs. Risiko}: bei vielen Entscheidungen ist eine Abwägung zwischen Ertrag und Risiko zu treffen.
	\item \textbf{Aggregation durch Märkte}: Wertpapiermärkte aggregieren Präferenzen und Informationen
	\item \textbf{Arbitragefreiheit}: Preise an kompetativen Wertpapiermärkten zeichnen sich durch die Abwesenheit von Arbitrage aus
\end{itemize}

\section{Relative vs. absolute Bewertung}
Es gibt zwei teilweise gegenläufige Ansätze zur Bewertung von Assets
\begin{description}
	\item \textbf{absolut}: mit Bezug zu grundlegenden makroökonomischen Faktoren.
		\begin{itemize}
			\item Konsum-basierte oder allgemeine Gleichgewichtsmodelle
			\item Beispiel: capital Asset Pricing Model
		\end{itemize}
	\item \textbf{relativ}: zu gegebenen anderen Wertpapieren (Basiswertpapieren)
		\begin{itemize}
			\item Duplikatopn und Gesetz des einen Preises
			\item Beispiel: Optionspreistheorie nach Black/Scholes
		\end{itemize}
\end{description}
Typische Anwendungen enthalten beide Aspekte und der Anteil zu welchem man den einen oder anderen Ansatz  wählt, hängt für gewöhnlich vom betrachteten Asset und Grund für die Bewertung ab.

\section{Bewertungsprinzip}
Asset Pricing liegt eine einfach Grundidee zugrunde:
$$ \textit{Preise entsprechen erwarteten diskontierten Zahlungen} $$ \smallskip

\textbf{Zahlungsstrom}: Zukünftige unsichere Zahlungen 
\begin{itemize}
	\item Zeit- und Risikodimension
\end{itemize}
Bewertungsprinzip berücksichtigt Dimension durch geeignete
\begin{itemize}
	\item Diskontierung
	\item Erwartungsbildung
\end{itemize}

\chapter{Stochastischer Diskont-Faktor Ansatz}

\enquote{Asset pricing theory all stems from one simple concept, presented in the first page of the first chapter of this book: \textbf{price equals expected dicounted payoff}. The rest is elaboration, special cases, and a closet full of tricks that make the central equation useful from one or another application.} ~\\
	\begin{tabular}{lr}
  		\hspace{6.35cm} & \textit{Quelle: John H. Cochrane, Asset Pricing, S. xiii}
	\end{tabular} ~\medskip

Fragestellung: Gegebener Zahlungsstrom (z.B. Dividenden, Zinsen, Auszahlungen von Derivaten, Rückflüsse aus Investitionen, $\dotsc$)
\begin{itemize}
	\item Welchen Wert besitzt der Zahlungsstrom?
	\item Wie wirken Zeit und Risiko
	\item Wie ändert sich Wert, wenn sich Ökonomie verändert? (Risikomanagement)
\end{itemize}
Einfacher formaler Rahmen:
\begin{itemize}
	\item zukünftige (unsichere) Zahlung $x_{t+1}$
	\item gesucht: Heutiger Wert $p_t$
\end{itemize} ~\newpage

\section{Präferenzen der Investoren}

Idee: Investor trifft wirtschaftliche Entscheidungen mit dem Ziel, möglichst günstigen Konsumstrom zu erreichen. ~\bigskip

Formal über Nutzenfunktionen:
	$$ U(c_t, c_{t+1}) = u(c_t) + \beta \cdot \mathbb{E}_t\big[ u\left(c_{t+1}\right) \big] $$
\begin{description}
	\item $c_t =$ Konsum in $t$
	\item $c_{t+1} =$ Konsum in $t+1$ 
\end{description}

Plausible Annahmen bezüglich Eigenschaften von Nutzenfunktionen

\begin{description}
	\item \textit{positiver Grenznutzen}: Nutzen wächst, wenn Konsum in beliebigem Zeitpunkt wächst, d.h. \enquote{mehr ist besser als weniger} (nicht gesättigte Investoren)
	\item \textit{abnehmender Grenznutzen}: Je höher der Konsum in einem Zeitpunkt, desto geringer ist der durch zusätzliche Konsumeinheit erzeugte Nutzenzuwachs
\end{description} % todo grafische Darstellung der Nutzenfunktion

Nutzenfunktion bildet Ungeduld und Risikoaversion der Investoren ab:

\begin{description}
	\item \textit{Ungeduld}: $\beta < 1$ erfasst Präferenz für frühere Zahlungen subjektive Diskontierung
	\item \textit{Risikoaversion}: zukünftiger Konsum $c_{t+1}$ unsicher, daher $\mathbb{E}_t \big[ u(c_{t+1}) \big]$. Krümmung der Nutzenfunktion $u$ zentral. ~\\
		Beispiel: 50/50 Wetter
		$$ \mathbb{E} \big[ u(c) \big] = 0.5 \cdot u ( \overline{c} + x ) + 0.5 \cdot u ( \overline{c} - x ) $$
		$u$ konkav $\Rightarrow$ Wette vermeiden
\end{description} % todo grafische Darstellung der 50%50 Wette

Gesamtnutzenfunktion $U(c_t, c_{t+1})$ am einfachsten über Nutzenindifferenzkurven darstellbar: % todo grafische Darstellung

\begin{itemize}
	\item Alle Punkte auf einer Kurve weisen denselben Nutzen auf.
	\item Je weiter entfernt vom Ursprung, desto höher der Nutzen $\rightarrow$ folgt aus positivem Grenznutzen
	\item Iso-Nutzenlinien verlaufen (streng) konvex. $\rightarrow$ folgt aus abnehmendem Grenznutzen
	\item Aversion gegen intertemporale Substitution
\end{itemize}

\section{Beispiele für Nutzenfunktionen}

\textbf{Power-Nutzenfunktion}
	$$ u(c) = \frac{c^{1-\gamma} - 1}{1 - \gamma} $$
\begin{itemize}
	\item $- c \frac{u''(c)}{u'(c)} = \gamma$: Konstante relative Risikoaversion
	\item strebt für $\gamma \rightarrow 1$ gegen Log-Nutzenfunktion
		$$ u(c) = \ln(c) $$
\end{itemize}

\textbf{Quadratische Nutzenfunktion}
	$$ u(c) = - \frac{1}{2} \left( \overline{c} - c \right)^2, \quad c < \overline{c} $$

\section{Zentrale Bewertungsbeziegung}

\begin{itemize}
	\item Investor konsumiert $c_t, c_{t+1}$ und kann beliebigen Anteil $\xi$ der Zahlung $x_{t+1}$ kaufen oder verkaufen
	\item Kalkül des Investors:
		$$ \max_{\xi} u(c_t) + \beta \mathbb{E}_t \big[ u(c_{t+1}) \big] \text{ u.d.N. } $$
		\begin{align*}
			c_{t} & = e_t - p_t \xi \\
			c_{t+1} & = e_{t+1} + x_{t+1} \xi
		\end{align*}
	\item Bedingung erster Ordnung:
		$$ p_t u'(c_t) = \mathbb{E}_t \big[ \beta u'(c_{t+1}) x_{t+1} \big] $$
	\item Investor kauft/verkauft solange bis Grenzosten = Grenzertrag
\end{itemize} \newpage

Aus der first-order Bedingung folgt 
\begin{itemize}
	\item zentrale Bewertungsgleichung
	$$ p_t = \mathbb{E}_t \big[ \beta \frac{u'(c_{t+1})}{u'(c_t)} x_{t+1} \big] $$
	\item in vielen Fällen hilfreich, obgleich sowohl \\
		\textbf{Preis} als auch \textbf{Konsum} endogene Größen!
\end{itemize}

\section{Stochastischer Diskontfaktor}

Hilfreiche Separation:

\begin{itemize}
	\item Mit stochastischem Diskontfaktor
	 $$ m_{t+1} = \beta \frac{u'(c_{t+1})}{u'(c_t)} $$
	\item vereinfacht sich zentrale Bewertungsbeziehung zu
	 $$ p_t = \mathbb{E}_t \big[ m_{t+1} x_{t+1} \big] $$
\end{itemize}

Stochastische Diskontfaktor verallgemeinert übliches Verständnis von Diskontfaktoren:

\begin{itemize}
	\item Sichere Zahlung $x_{t+1}$: $p_t = \frac{1}{R^f} x_{t+1} = \frac{1}{1+r^f} x_{t+1}$
	\item Risikoadjustierte Diskontierung: $p_t = \frac{1}{ER^i} \mathbb{E}_t \big[ x_{t+1} \big]$
\end{itemize}

Beachte
\begin{itemize}
	\item stochastischer Diskontfaktor $m_{t+1}$
		\begin{itemize}
			\item ist zufällig
			\item für alle Assets (bzw. Cash Flows $x_{t+1}$) identisch
		\end{itemize}
	\item Wie passt das mit der üblichen Vorstellung zusammen, dass riskantere Titel eine höhere Diskontierung erfordern?
\end{itemize}

Interpretationen und alternative Bezeichnungen von $m_{t+1}$
\begin{itemize}
	\item Grenzrate der Substitution: $m_{t+1} = \beta \frac{u'(c_{t+1})}{u'(c_t)}$
	\item Pricing Kernel, Dichte der Zustandspreise
\end{itemize}

\section{Beispiele für Preise und Zahlungen}

\begin{enumerate}[label=\arabic*\upshape.]
	\item \textbf{Aktieninvestment}: $p_t$: Preis in $t$, $x_{t+1} = p_{t+1} + d_{t+1}$: Zahlung in $t+1$ mit Dividendenzahlung $d_{t+1}$ in $t+1$; dann gilt $p_t = \mathbb{E}_t \big[ m ( p_{t+1} + d_{t+1} ) \big]$
	\item \textbf{Brutto-Return}: Interpretiere $R_{t+1} = \frac{x_{t+1}}{p_t}$ als Payoff in $t + 1$ mit Preis 1 dann gilt $1 = \mathbb{E}_t \big[ mR \big]$
	\item \textbf{Überschuss-Return}: als Zahlung $R_{t+1}^e = R_{t+1}^a - R_{t+1}^b$ in $t + 1$ eines Portfolios ohne Kapitaleinsatz, d.h. $p_t = 0$ dann gilt $0 = \mathbb{E}_t \big[ mR^e \big]$
	\item \textbf{Einperiodige Anleihe}: $p_t$: Anleihepreis in $t$, Rückzahlung $x_{t+1} = 1$, dann gilt $p_t = \mathbb{E}_t \big[ m \big]$
	\item \textbf{Geldmarktkonto}: $p_t = 1$, Rückfluss in $t + 1$: $R^f = (1 + r^f)$ dann gilt $1 = \mathbb{E}_t \big[ mR^f \big]$
	\item \textbf{Kaufoption}: $p_t = C$, $x_{t+1} = \max(S_{t+1} - K, 0)$, dann gilt $C = \mathbb{E}_t \big[ m \left( \max ( S_{t+1} - K, 0 ) \right) \big]$
\end{enumerate}

\chapter{Klassische Theorien}

Durch einfache Umformungen der zentralen Bewertungsbeziehung
$$ p = \mathbb{E}[mx] $$
lassen sich viele finanzwirtschaftliche Theorien und Konzepte leicht ableiten:
\begin{enumerate}[label=\arabic*\upshape.]
	\item \textbf{Ökonomie der Zinsen}: Wann und warum sind Zinsen hoch oder niedrig?
	\item \textbf{Risikoanpassung}: Wovon hängt die Risikoanpassung ab?
	\item \textbf{Unsystematisches Risiko}: Warum wird unsystematisches Risiko nicht vergütet?
	\item \textbf{Beta als Risikomaß}: Welche Beziehung besteht zwischen erwarteten Renditen und Beta?
	\item \textbf{$\mu$-$\sigma$-Rand}: Welche Rendite/Risiko-Kombinationen sind erreichbar?
	\item \textbf{Equity Premium Puzzle}: Warum sind Risikoprämien von Aktien so hoch?
\end{enumerate}

\section{Ökonomie der Zinsen}

Interpretation der risikolosen Verzinsung $R^f = 1 + r^f$.
\begin{itemize}
	\item Aus zentraler Bewertungsbeziehung folgt für das Geldmarktkonto:
		$$ 1 = \mathbb{E} \left[ m R^f \right] \Rightarrow R^f = \frac{1}{\mathbb{E}[m]} $$
	\item Bei Sicherheit folgt für eine isoelastische Nutzenfunktion
		$$ u(c) = \frac{c^{1-\gamma} - 1}{1 - \gamma}: m_{t+1} = \beta \frac{u'(c_{t+1})}{u'(c_{t})} = \beta \left( \frac{c_{t+1}}{c_t} \right)^{-\gamma} $$
		$$ \Rightarrow R^f = \frac{1}{\beta} \left( \frac{c_{t+1}}{c_t} \right)^{\gamma} $$
	\item Somit gilt:
		\begin{itemize}
			\item Realzinsen sind hoch, wenn Investoren ungeduldig sind (niedriges $\beta$)
			\item Realzinsen sind hoch, wenn das Konsumwachstum hoch ist.
			\item Realzinsen reagieren sensitiver auf Änderungen des Konsumwachstums bei hoher Riskoaversion (hohes $\gamma$).
		\end{itemize}
\end{itemize}
	
Unter Annahme von Unsicherheit:	
\begin{itemize}
	\item Mit $\beta = e^{-\delta}$ und $\Delta c_{t+1} = \ln \left( \frac{c_{t+1}}{c_t} \right)$ folgt
		$$ m_{t+1} = e^{-\delta} e^{-\gamma \Delta c_{t+1}} \approx 1 - \delta - \gamma \Delta c_{t+1} $$
	\item Somit
		$$ R^f = \frac{1}{\mathbb{E}[m_{t+1}]} \approx \frac{1}{1 - \delta - \gamma \mathbb{E}_t[\Delta c_{t+1}]} \approx 1 + \delta + \gamma \mathbb{E}[\Delta c_{t+1}] $$
	\item Grundsätzlich identische Implikationen wie im deterministischen Fall:
		$$ R^f \approx 1 + \delta + \gamma \mathbb{E} [\Delta c_{t+1}] $$
\end{itemize}
Wie schon zuvor sind Zinsen hoch
\begin{itemize}
			\item bei sehr ungeduldigen Investoren (niedriges $\beta$ bzw. hohes $\delta$)
			\item bei hohem erwarteten Konsumwachstum $\mathbb{E}_t[\Delta c_{t+1}]$
				\begin{itemize}
					\item wer weiß, dass er in Zukunft reicher sein wird, braucht hohe Zinsen, damit er bereit ist, heute auf Konsum zu verzichten und dafür zu sparen
					\item Zinsen sind höher in Aufschwungphasen als in Rezessionen
				\end{itemize}
\end{itemize}
Wie schon zuvor
\begin{itemize}
	\item Sensitivität bzgl. Konsumwachstum nimmt mit Risikoaversion $\gamma$ zu
		\begin{itemize}
			\item Beachte: Hohes $\mathbb{E}_t[\Delta c_{t+1}]$ (Aufschwung), hohes $R^f$; ~\\
				niedriges $\mathbb{E}_t[\Delta c_{t+1}]$ (Abschwung), niedriges $R^f$;
			\item Stärke der Veränderung (ob positiv  oder negativ) steigt mit $\gamma$
		\end{itemize}
\end{itemize}

Nun zu Aspekt des Risikos. ~\\
Betrachte hierzu Approximation zweiter Ordnung:
	$$	R^f \approx 1 + \delta + \gamma \mathbb{E}_t[\Delta c_{t+1}] - \frac{1}{2} \gamma^2 \sigma^2_t (\Delta c_{t+1}) $$
Höhere Volatilität des Konsumwachstums (hohes $\sigma$)
\begin{itemize}
	\item führt zu niedrigeren Zinsen
		\begin{itemize}
			\item in unsicheren Zeiten spart man lieber vorsorglich
			\item hohe Sparnachfrage reduziert Zinsen
		\end{itemize}
\end{itemize}

Umgekehrt
\begin{itemize}
	\item ist Konsumwachstum hoch, falls Zinsen hoch sind (bei hohen Zinsen wird mehr gespart)
	\item ist Konsum weniger sensitiv bzgl. Zinsänderungen, wenn $\gamma$ hoch ist (hohes $\gamma \Rightarrow$ starker Wunsch nach gleichmäßigem Konsumstrom)
\end{itemize}

Was determiniert wen?
\begin{itemize}
	\item Konsum determiniert Zinsen
	\item Zinsen determinieren Konsum
\end{itemize}

Beachte: ~\\
Bei der Power-Nutzenfunktion gilt: Krümmungsparameter $\gamma$ steuert gleichzeitig
\begin{itemize}
	\item intertemporale Substitution: Aversion gegenüber zeitlich schwankenden Konsummöglichkeiten
	\item  Risikoaversion: Aversion gegenüber Veränderungen der Konsummöglichkeiten durch unterschiedliche Zustände
	\item vorsorgliche Ersparnis: abhängig von der dritten Ableitung der Nutzenfunktion
\end{itemize}
 Allgemeinere Nutzenfunktionen entkoppeln die drei Einflüsse. Beispiel: ~ Rekursive Nutzenfunktion (Epstein-Zin Nutzenfunktion) 

\section{Risikoanpassung}

\begin{itemize}
	\item Aus der Definition $\operatorname{cov}(m,x) = \mathbb{E}[mx] -  \mathbb{E}[m]  \mathbb{E}[x]$ folgt
		$$ p =  \mathbb{E}[mx] =  \mathbb{E}[m] \mathbb{E}[x] - \operatorname{cov}(m, x) $$
	\item Mit $R^f = \frac{1}{\mathbb{E}[m]}$ folgt
		$$ p = \frac{\mathbb{E}[x]}{R^f} + \operatorname{cov}(m, x) $$
\end{itemize}

Preis ergibt sich aus
\begin{itemize}
	\item Diskontierung des erwarteten Payoffs mit risikolosem Zinssatz
		\begin{itemize}
			\item Standardbarwert-Kalkül bei Sicherheit bzw. Risikoneutralität
			\item Aspekt \enquote{Zeit}
		\end{itemize}
	\item Risikokorrektur über Kovarianzterm
		\begin{itemize}
			\item je stärker Kovarianz mit Diskontfaktor $m$, desto höher der Preis
			\item Aspekt \enquote{Risiko}
		\end{itemize}
\end{itemize}

\subsubsection{Wirkungsweise der Risikoanpassung}

Mit $m = \beta \frac{u'(c_{t+1})}{u'(c_t)}$ folgt
	$$	p = \frac{\mathbb{E}[x]}{R^f} + \frac{\operatorname{cov}(\beta u'(c_{t+1}), x_{t+1})}{u'(c_t)} $$
Risikoanpassung
\begin{itemize}
	\item verringert Preis bei positiver Kovarianz mit Konsum ($u'(c)$ sinkt in $c$)
	\item erhöht Preis bei negativer Kovarianz mit Konsum
\end{itemize}
Warum?

\begin{beispiel}
	Betrachte zwei Wertpapiere mit 
	
	\begin{figure*}[h!] \centering
		\begin{tabular}{l|cc}
  			~  & gute Zeiten (0.5) & schlechte Zeiten (0.5) \\
  			\hline
 			$WP_1$ & $100$ & $0$ \\
  			$WP_2$ & $0$ & $100$
		\end{tabular}
	\end{figure*} 


	Für welches Wertpapier sind Sie bereit mehr zu bezahlen?	
\end{beispiel}


Ergebnis:
\begin{itemize}
	\item Für Assets, die mehr zur Konsumglättung beitrag, werden höhere Preise bezahlt.
	\item Bei unsicherem Payoff mit gegebenem Erwartungswert $\mathbb{E}[x]$
		\begin{itemize}
			\item wird Preis nach unten korrigiert, falls Payoff in schlechten Zeiten niedrig ist
			\item wird Preis nach oben korrigiert, falls Payoff in schlechten Zeiten hoch ist (Versicherungsidee)
		\end{itemize}
\end{itemize}

Rolle der Risikoaversion
\begin{itemize}
	\item Betrachte wieder isoelastische Nutzenfunktion $u(c) = \frac{c^{1-\gamma} - 1}{1 - \gamma}$
	\item Höheres $\gamma \Rightarrow$ stärkere Risikokorrektur
	\item Formal: Aus Approximation $m_{t+1} \approx 1 - \delta - \gamma \Delta c_{t+1}$ folgt 
		$$ \operatorname{cov}(m_{t+1}, x_{t+1}) \approx - \gamma \operatorname{cov}(\Delta c_{t+1}, x_{t+1}) $$
		und damit
		$$	p_t \approx \frac{\mathbb{E}_t[x_{t+1}]}{R^f} - \gamma \operatorname{cov}(\Delta c_{t+1}, x_{t+1}) $$
\end{itemize}

Warum zählt die Kovarianz und nicht Varianz der Zahlung? \medskip

Die eigentliche Frage ist nämlich, die Schwankung im resultierenden Nutzenstorms und nicht eines einzelnen Preises.

\begin{itemize}
	\item Investor interessiert sich nicht für Volatilität eines einzelnen Wertpapiers sondern für resultierenden Konsum und ggf. dessen Varianz
	\item $\sigma^2(c + \xi x) = \sigma^2(c) + 2 \xi \operatorname{cov} (c x) + \xi^2 \sigma^2(x)$ ~\\
		Hier ist der mittlerer Term von erster Ordnung; da wird beim hinteren Term $\xi^2$ haben, können wir bei marginalen Betrachtungen diesen Fallen lassen (fordert: ex-post Betrachtung), das Kalkül gilt hier also immer!. Ex-ante können wir das analoge Argument aufbringen, falls man in marginale Beiträge ($\xi$) investieren kann und keine z.B. all-or-nothing Situation hat.
	\item Vorstellung: Portfolios und damit auch $c$ und $m$ bereits angepasst
	\item Welchen Beitrag hat letzte marginale Einheit (sehr kleines $\xi$) von $x$?
\end{itemize}

Betrachte nun Returns verschiedener Wertpapiere $i, j$
	$$ R^i = \frac{x_{t+1}^i}{p_t^i}, ~R^j = \frac{x_{t+1}^j}{p_t^j} $$
Dann gilt
$$ 1= \mathbb{E} \left[ m R^i \right] = \mathbb{E} \left[ m R^j \right] $$
\begin{itemize}
	\item Obgleich erwartete Returns i.d.R. verschieden, sind erwartete diskontierte Returns immer 1!
	\item Weiter gilt $1 = \mathbb{E}[m] \mathbb{E}[R^i] + \operatorname{cov}(m, R^i)$
		$$ \iff R^f = \frac{1}{E(m)} = \mathbb{E}(R^i) + \frac{\operatorname{cov}(m, R^i)}{\mathbb{E}(m)} $$
		$$ \Rightarrow \mathbb{E}[R^i] - R^f = - R^f \operatorname{cov}(m,R^i) $$ % todo Nachrechnen
	\item $m = \beta u'(c_{t+1}) /u'(c_t)$ eingesetzt liefert
		$$ \mathbb{E}[R^i] - R^f = -\frac{\operatorname{cov}(u'(c_{t+1}), R^i)}{\mathbb{E}[u'(c_{t+1}]} $$
\end{itemize}

Aus Überrenditendarstellung
		$$ \mathbb{E}[R^i] - R^f = -\frac{\operatorname{cov}(u'(c_{t+1}), R^i)}{\mathbb{E}[u'(c_{t+1}]} $$
folgt:
\begin{itemize}
	\item die erwartete Rendite jedes Wertpapiers entspricht der risikolosen Verzinsung zuzüglich einer Risikokorrektur
	\item Wertpapiere, deren Returns positiv mit Konsum variieren führen zu volatilerem Konsum und müssen daher höhere erwartete Returns liefern. $\Rightarrow \mathbb{E}(R^i) > R^f$
	\item Umgekehrt können Wertpapiere, deren Returns negativ mit dem Konsum variieren und damit Konsum glätten (z.B. Versicherungen) niedrigere erwartete Returns bieten. $\mathbb{E}(R^i) < R^F$ (vgl. Wertpapiere von letzter Woche denen Preise höher als 50 gegeben wurden).
\end{itemize}

Beachte nochmal

\begin{itemize}
	\item Übliche Vorstellung
		$$ p^i = \frac{\mathbb{E}[x_{t+1}^i]}{ER^i} $$
		mit Diskontfaktor $\frac{1}{ER^i}$ etwa aus CAPM, der wertpapierspezifisch ist!
	\item Hier
		$$ p^i = \mathbb{E}[m x_{t+1}^i] $$
		mit stochastischemm Diskontfaktor $M$ (innerhalb des Erwartungswertes!), der für alle Wertpapiere identisch ist!
	\item Wie passt das zusammen? Der Dikontfaktor ist auch wertpapierspezifisch. Dies sieht man, falls man den stochastischen Diskontfaktor aus dem Erwartungswert rausziehen, denn dann taucht dieser in der Kovarianz auf.
\end{itemize}

\section{Unsystematisches Risiko}

Aus 
$$ p = \mathbb{E}[mx] = \mathbb{E}[m] \mathbb{E} [x] + \operatorname{cov}(m,x)$$
foolgt unmittelbar
$$ p = \frac{\mathbb{E}[x]}{R^f} \text{ für } \operatorname{cov}(m, x) = 0 $$
\begin{itemize}
	\item mit dem Diskontfaktor $m$ unkorrelierte Zahlungen erfordern keine Risikokorrektur im Preis
	\item solches unsystematisches Risiko wird folglich nicht vergütet
	\item erwartete Rendite entspricht der riskolosen Rendite
\end{itemize}
Beachte: Ergebnis gilt unabhängig
\begin{itemize}
	\item von $\sigma^2(x)$, d.h. wie volatil die Zahlung ist
	\item vom Ausmaß der Risikoaversion
\end{itemize}

Idee: Zerlege $x$ in
\begin{description}
	\item \textbf{systematische komponente}: Mit Diskontfaktor $m$ perfekt korrelierte Komponente $\operatorname{proj}(x|m)$
	\item \textbf{unsystematiscehe Komponente}: Zu Diskontfaktor orthogonale Komponente $\epsilon{e}$
\end{description}

$$ {\color{gray} x = \operatorname{proj}(x|m) + \epsilon \Rightarrow b = \frac{\mathbb{E}(mx)}{\mathbb{E}(m^2)} } $$

Intuitiv:
\begin{itemize}
	\item Lineare Regression ohne Konstante $x = bm + \epsilon$ führt auf mit $m$ perfekt korrelierte Komponente $bm$ und Restgröße $\epsilon$ mit $E[m\epsilon] = 0$
	\item Opffensichtlich gilt
		\begin{itemize}
			\item Preis von $\epsilon$: $p(\epsilon) = \mathbb{E}[m\epsilon] = 0$
			\item Preis von $\operatorname{proj}(x, m) \colon p(\operatorname{proj}(x, m)) = \mathbb{E}[xm] = p(x)$ $$= p(b m) = \mathbb{E}[m bm] = \mathbb{E} \left[ \frac{\mathbb{E}(mx)}{\mathbb{E}[m^2]} m^2 \right] = \mathbb{E}(mx)$$
		\end{itemize}
\end{itemize}

\section{Beta als Risikomaß}

$$
\textbf{ $\beta_i$: Sensitivität der Rendite von Wertpapier i gegenüber der Rendite des ganzen Marktes}$$
\begin{itemize}
	\item  das klassische Risikomaß der Finanzwirtschaft
	\item typischerweise anhand des CAPM  bestimmt
	\item Anwendung als Maß für systematisches Risiko
\end{itemize}
 
% todo Zeichnung

formal: $\beta_i = \frac{\operatorname{cov}(R^i, R^M)}{\operatorname{var}(R^M)}$ mit $R^M$ als Marktrendite. ~\\

Umformung der Return-Beziehung $\mathbb{E}[R^i] ) R^f - R^f \operatorname{cov}(m, R^i)$ führt zu

% Lecture Notes - End 
% Excercise - Start

\appendix
\chapter{Übungen}

\subsubsection*{Aufgabe 7}

\begin{enumerate}
	\item Es gilt (vgl. letzte Übung): $m_{t+1} = \beta \frac{u'(c_{t+1})}{u'(c_{t})}$.
		\begin{itemize}
			\item Im schlechten Zustand nimmt der stochastische Diskontfaktor relativ große Werte an (Grenzwert des Konsums ist hoch).
			\item Im guten Zustand relativ geringe Werte.
		\end{itemize}
		Damit ist:
		\begin{align*}
			& m_{t+1} = 0.97 \text{: ungünstiger Zustand} \\
			& m_{t+1} = 0.85 \text{: günstiger Zustand}
		\end{align*}
		\textit{Beachte: \enquote{Im schlechten Zustand} bezieht sich eigentlich relativ auf den Vergleich zu vorhergehenden Periode.}
	\item Eine Power-Nutzenfunktion hat bei uns die allgemeine Form:
		$$ u(c) = \frac{c^{1-\gamma} - 1}{1 - \gamma} $$
		Daraus folgt $u'(c) = c^{-\gamma}$ und damit: % todo -gamma scheint mir im folgenden nicht zu sitmmen
		$$ m_{t+1} = \beta \frac{c_{t+1}^{\gamma}}{c_t^{-\gamma}} = \beta \left( \frac{c_t}{c_{t+1}} \right)^\gamma $$
		Ist $\gamma > 0 \Rightarrow$ je größer der Konsum in einem Zustand in $t+1$ ist, desto kleiner $m_{t+1}$ und umgekehrt
	\item $p_A > p_B$ da Wertpapier $A$ im ökonomisch schlechteren Zustand mehr auszahlt und der Erwartungswert der Auszahlung beider Wertpapiere gleich ist.
	\item Es ist
		\begin{align*}
			p_t^A & = \mathbb{E}(m_{t+1} x_{t+1}^A) = 0.5 \cdot 0.85 \cdot 5 + 0.5 \cdot 0.97 \cdot 8 = 6.01 \\
			p_t^B & = analog = 5.83
		\end{align*}
		% todo Zeichnung: Siehe Block
		$\mathbb{E}[R_{t+1}^B] > \mathbb{E}[R_{t+1}^A]$, $p_A > p_B$
	\item Es ist
		\begin{align*}
			p_t & = \mathbb{E}[m_{t+1} X_{t+1}] \\
				& = \mathbb{E}[m_{t+1}] \cdot \mathbb{E}[x_{t+1}] + \operatorname{cov}(m_{t+1}, x_{t+1}) \\
			& = \frac{\mathbb{E}[x_{t+1}]}{R^t} + \operatorname{cov}(m_{t+1}, x_{t+1})
		\end{align*}
		$\Rightarrow \mathbb{E}(x_{t+1}Â) = \mathbb{E}(x_{t+1}^B) = 6.5$
		$$ R^f = \frac{1}{\mathbb{E}(m_{t+1})} = \frac{1}{0.91} = 1.10 $$
		$\Rightarrow \frac{\mathbb{E}(x_{t+1}^A)}{R^f} = \frac{\mathbb{E}(x_{t+1}^B)}{R^f} = 5.91$
		\begin{align*}
			& \operatorname{cov}(m_{t+1}, x_{t+1}) = 0.5 \left( 0.85 - 0.91 \right) \left( 5 - 6.5 \right) + 0.5 \left(0.97 - 0.91 \right) \left( 8 - 6.5 \right) = 0.09 \\
			& \operatorname{cov}(m_{t+1}, x_{t+1}^B) = analog = -0.09 
		\end{align*}
		wobei $\operatorname{cov}(a, b) = \mathbb{E}[(a - \overline{a})(b - \overline{b})]$, mit $\overline{x} = \mathbb{E}[x]$, 
	$$\Rightarrow p^A > p^B$$
	\item Da die $\operatorname{cov}(\cdot, \cdot)$ negative sowie positive Werte annehmen kann, ergibt sich bei geeiegneten Werten für $\mathbb{E}(x_{t+1})$ und $R^f$ für manche Wertpapiere auch bei $\mathbb{E}(x_{t+1}) < 0$ ein positiven Preis. Beispiel: Versicherung
\end{enumerate}

\newpage

\subsubsection*{Aufgabe 8 (Der $\mu$-$\sigma$-Rand)}

Betrachten Sie ein Ökonomie mit zwei Zeitpunkten ($t$ und $t+1$) und einem risikolosen Zinssatz von 10\%.

\begin{enumerate}
	\item Unterstellen Sie für den SDF, das $m_{t+1} = a + b R^{mv}$, wobei $a$ und $b$ Parameterwerte und $R^{mv}$ die Rendite eines Wertpapiers darstellt.
		\begin{enumerate}
			\item Welche Voraussetzung muss erfüllt sein, damit dieser Zusammenhang gelten kann? Welche Implikation für die Bewertung von Wertpapieren enthält diese Annahme.
				\begin{proof}
					Damit der Zusammenhang $m_{t+1} = a + b R^{mv}$ gilt, muss die Rendite des Wertpapiers, $R^{mv}$ perfekt mit $m_{t+1}$ korreliert sein. ~\\
					Das impliziert, dass in $R^{mv}$ alle bewertungsrelevanten Informationen enthalten sein müssen.
				\end{proof}
			\item Maximale Sharpratio und
				\begin{proof}
					$p_{t+1} = \mathbb{E}(m_{t+1} x_{t+1}) \iff 1 = \mathbb{E}(m_{t+1} R_{t+1})$
					\begin{align*}
						& \iff 1 = \mathbb{E}(m_{t+1}) \mathbb{E}(R_{t+1}) + \operatorname{cov}(m_{t+1}, R_{t+1}) \qquad \qquad \qquad \\
						& \iff \mathbb{E}(R_{t+1}) = R^f - \frac{\operatorname{cov}(m_{t+1}, R_{t+1})}{\mathbb{E}(m_{t+1})}  \\
						& \iff \mathbb{E}(R_{t+1}) - R^f = - \rho_{m_{t+1}, R_{t+1}} \sigma_{R_{t+1}} \frac{\sigma_{m_{t+1}}}{\mathbb{E}(m_{t+1})} \\
						& \iff \frac{\mathbb{E}(R_{t+1}) - R^f}{\sigma_{R_{t+1}}} = -\rho_{m_{t+1}, R_{t+1}} \frac{\sigma_{m_{t+1}}}{\mathbb{E}(m_{t+1})} 
					\end{align*}
					
					Mit $|\rho| \leq 1$ gilt:
					$$ \left| \frac{\mathbb{E}(R_{t+1}) - R^f}{\sigma_{R_{t+1}}} \right| \leq \frac{\sigma_{m_{t+1}}}{\mathbb{E}(m_{t+1})}$$
					Aus $m_{t+1} = a + b \cdot R^{mv}_{t+1}$ folgt:
					$$
					  \begin{rcases}
						\operatorname{var}(m_{t+1}) & = b^2 \operatorname{var}(R_{t+1}^{mv}) \\
						\sigma_{m_{t+1}} & = \sqrt{b^2 \operatorname{var}(R_{t+1}^{mv})} = \sqrt{2} \\
						\frac{1}{\mathbb{E}(m_{t+1})}& = R^f = 1.1
					  \end{rcases} \Rightarrow \frac{\sigma_{m_{t+1}}}{\mathbb{E}(m_{t+1})} = 1.55 
					$$
				\end{proof}
		\end{enumerate}
\end{enumerate}

\subsubsection*{Aufgabe 9 (Der $\mu$-$\sigma$-Rand und die Beta-Darstellung)}

Gegeben ist eine Ökonomie mit zwei Zeitpunkten ($t$ und $t+1$) mit einem risikolosen Zinssatz in Höhe von $R^f$, sowie ein effizienter Rand-Return $R^{mv}$. Nehmen Sie an

\begin{enumerate}
	\item Es gilt $m_{t+1} = a + b \cdot R^{mv}$
		\begin{align*}
			p_t & = \mathbb{E}(m_{t+1} x_{t+1}) = \mathbb{E}(m_{t+1}) \mathbb{E}(x_{t+1}) + \operatorname{cov}(m_{t+1}, x_{t+1}) \\
			& \iff p_{t} = \frac{\mathbb{E}(x_{t+1})}{R^f} + \operatorname{cov}(m_{t+1}, x_{t+1}) \\
			& \iff p_{t} = \frac{\mathbb{E}(x_{t+1})}{R^f} + \operatorname{cov}(a + b R^{mv}, x_{t+1})\\
			& \iff p_{t} = \frac{\mathbb{E}(x_{t+1})}{R^f} + b \operatorname{cov}(R^{mv}, x_{t+1})
		\end{align*}
		$\Rightarrow$ falls $b$ bekannt können mit Hilfe von $R^{mv}$ alle Wertpapiere bewertet werden.
	\item Gilt für alle Wertpapiere auf dem $\mu$-$\sigma$-Rand außer dem risikolosen Instrument.
	\item Es gilt die folgenden Möglichkeiten
		\begin{itemize}
			\item 1. Möglichkeit: 
					\begin{align*}
					1 & = \mathbb{E}(m_{t+1} R^{mv}) = \mathbb{E}\left((a + b R^{mv}) R^{mv} \right) \\
					& = \mathbb{E}\left( a R^{mv} + b \left(R^{mv} \right)^2 \right)   \tag*{$(*)$} ~\\
					1 & = \mathbb{E}(m_{t+1} R^{f}) = \mathbb{E}\left((a + b R^{mv}) R^{f} \right) \\
					& = \mathbb{E}\left( a R^{f} + b R^{mv} R^f \right)   \tag*{$(**)$}
				\end{align*}
				Aus $(*)$: $\Rightarrow 1 = a \cdot \mathbb{E}(R^{mv}) + b \mathbb{E}\left(\left(R^{mv} \right)^2 \right)$ ~\\
				Aus $(**)$: $\Rightarrow 1 = a \cdot \mathbb{E}(R^{f}) + b \mathbb{E}\left(\left(R^{f} \right)^2 \right)$
				$$\Rightarrow$$
			\item 2. Möglichkeit: aus $|\rho| = 1$ folgt ($\mathbb{E}(m_{t+1}) = a + b \mathbb{E}(R^{mv})$)
				\begin{align*}
					m_{t+1} & = \mathbb{E}(m_{t+1}) + b \left( R^{mv} - \mathbb{E}(R^{mv}) \right) \tag*{$(+)$} \\
						& \iff m_{t+1} = \frac{1}{R^f} + b \left( R^{mv} - \mathbb{E}(R^{mv}) \right)  \tag*{$(***)$}
				\end{align*}
				Außerdem: $1 = \mathbb{E}(m_{t+1} E^{mv}$. Daraus folgt indem wir $(***)$ einsetzen:
					\begin{align*}
						1 & = \mathbb{E} \left[ \left( \frac{1}{R^f} + b \left( R^{mv} - \mathbb{E}_t \left( R^{mv} \right) \right) \right) R^{mv} \right] \\
						& \iff 1 = \frac{1}{R^f} \mathbb{E}(R^{mv}) + b \mathbb{E}\left( \left((R^{mv} \right)^2 \right) - b \mathbb{E} \left(R^{mv} \right)^2 \\
						& \iff 1 = \frac{1}{R^f} \mathbb{E}(R^{mv}) + b \operatorname{var}(R^{mv}) \\
						& \iff b = - \frac{\mathbb{E}(R^{mv}) - R^f}{R^f \operatorname{var}(R^{mv})}
					\end{align*} 
					In $(+)$ einsetzen:
					\begin{align*}
						m_{t+1} & = \mathbb{E}(m_{t+1}) + \left( - \frac{\mathbb{E}(R^{mv}) - R^f}{R^f \operatorname{var}(R^{mv})} \right) \left( R^{mv} - \mathbb{E}(R^{mv} \right) \\
						& \iff m_{t+1} = \underbrace{\frac{1}{R^f} + \mathbb{E}(R^{mv}) \frac{\mathbb{E}(R^{mv} - R^f}{R^f \operatorname{var}(R^{mv})}}_{\eqqcolon a} \underbrace{- \frac{\mathbb{E}(R^{mv}) - R^f}{R^f \operatorname{var}(R^{mv})}}_{\eqqcolon b} R^{mv} \tag*{$(++)$} \\
						& \iff m_{t+1} = a + b R^{mv}
					\end{align*} 
		\end{itemize}
		\item Beta Darstellung
			$$ p_{t} = \mathbb{E}(m_{t+1} x_{t+1}) \iff 1 = \mathbb{E}(m_{t+1}) \mathbb{E}(R^i_{t+1}) + \operatorname{cov}(m_{t+1}, R^i_{t+1}) $$
			$$ \mathbb{E}(R_{t+1}^i) = R^f - \frac{\operatorname{cov}(m_{t+1}, R_{t+1}^i)}{\mathbb{E}(m_{t+1})} $$
			Aus $(++)$ folgt
			$$ \mathbb{E}(R_{t+1}^i) = R^f + \frac{\mathbb{E}(R^{mv} - R^f)}{R^f \operatorname{var}(R^{mv})} \operatorname{cov}(R^{mv}, R_{t+1}^i) \frac{1}{\mathbb{E}(m_{t+1})} $$
			$$ \iff \mathbb{E}(R_{t+1}^i) = \R^f + \frac{\mathbb{E}(R^{mv} - R^f}{\operatorname{var}(R^{mv}} \operatorname{cov}(R^{mv}, R^i) $$
			$$ \iff \mathbb{E}(R^i_{t+1}) = R^{f} + \underbrace{\frac{\operatorname{cov}(R^{mv}, R^{i}_{t+1})}{\operatorname{var}(R^{mv})}}_{\eqqcolon \beta_{i, mv}} \underbrace{\left( \mathbb{E}(R^{mv} - R^f \right)}_{\eqqcolon \lambda_{mv}} $$
			$$ \iff \mathbb{E}(R^i_{t+1}) = R^f + \beta_{i, mv} \lambda_{mv} $$
		\item Welche Annahme trift man oft in der parktischen Umsetzung dieser Bewertungsbeziehung
			\begin{proof}
				$R^{mv} =$ Rendite des Marktportfolios (z.B. Dax 30)
			\end{proof}
\end{enumerate}

\newpage

\subsection*{Sonderübung}

\subsubsection*{Aufgabe 1}
Nehmen Sie Stellung zu folgenden Aussagen:
\begin{enumerate}
	\item Wertpapiere, deren Auszahlung positiv mit dem stochastischem Diskontfaktor  korrelieren, besitzen im Gleichgewicht höhere erwartete Renditen, die für das höhere Risiko solcher Wertpapiere kompensieren.
		\begin{proof}
			Die Aussage ist falsch. Wertpapiere deren erwarteten. Renditen positiv mit dem stoch. Diskontfaktor korrelieren, zahlen in Zuständen mit hohem stoch. Diskontfaktor wenig und in Zuständen mit niedrigem stoch. Disktontfaktor viel (es ist ja: SDF hoch = schlechter Zustand und vice versa). Solch ein Wertpapier besitzt einen Versicherungscharakter. Wichtig ist, dass die Renditen-Preisbeziehung invers ist. Im Gleichgewicht beizten sie also niedrige erwartete Renditen, da sie ein geringeres Risiko besitzen.
		\end{proof}
	\item Wertpapiere, die nicht auf dem $\mu$-$\sigma$-Rand liegen, sind nicht effizient und werden deshalb von den Investoren nicht nachgefragt.
		\begin{proof}
			Die Aussage ist falsch. Diese Wertpapiere sind nicht effizient, allerdings sind sie auch nicht perfekt mit stochastischen Diskontfaktor korreliert und weisen unsystematisches Risiko auf. Damit können diese Wertpapiere nachgefragt werden, jedoch nicht alleine.
		\end{proof}
	\item Das CAPM kann nicht stimmen, da tatsächlich die Korrelation von zukünftigen Zahlung mit dem Konsumwachstum entscheidend für den Preis eines Wertpapiers ist nachgefragt.
		\begin{proof}
			Diese Aussage ist falsch. CAMP ist ein Spezialfall des konsumbasierten Asset Pricing Modells. Als Faktormodell nutzt das CAMP eine Approximation des aggregieren Nutzenwacstums
			$$ m_{t+1} = \beta \frac{u'(c_{t+1})}{u'(c_{t})} = a + b' f_{t+1}. $$ 
			Das CAMP-Modell approximiert also diesen Faktor linear und  stellt damit (vereinfacht) die Korrelation dar.
		\end{proof}
	\item Existiert in einer Ökonomie mindestens für jeden Zustand ein Wertpapier, das $24$ im entsprechenden Zustand und $0$ in allen anderen Zuständen auszahlt, so ist auch die Verteilung des stochastischen Diskontfaktors bekannt und alle anderen Wertpapiere können basierend darauf bewertet werden.
		\begin{proof}
			Diese Aussage ist falsch. Obwohl die Auszahlungsmatrix linear unabhängig ist, reicht diese nicht zur Bestimmung der Verteilung des stoch. Diskontfaktors aus und somit können nicht nicht alle weiteren Wertpapiere darauf bewertet werden. Vgl.
			$$ p(x) = \sum_{s \in S} \pi(s) m(s) x(s) $$
			Somit ist der Preis und Eintrittswahrscheinlichkeit müssen gegeben sein.
		\end{proof}
	\item  Da die Zentralbank den Zinssatz festlegt, muss bei der empirischen Anwendung des konsumbasierten Ansatzes stets davon ausgegangen werden, dass der Preis des risikolosen Instruments exogen festgelegt wird. Die Diskussion inwiefern sich die Zinsen z.B. in Folge von geänderter Risikoaversion der Investoren ändern, ist deshalb lediglich exemplarischer Natur und nicht mit der Empirie vereinbar. 
		\begin{proof}
			Nur teilweise richtig. Wechselwirkung zwischen Konsum \% Zinsen
				$$ \Rightarrow \text{ Makro-Sicht } \Rightarrow \text{ Perspektive einer Einzelperson } $$
				Zentralbank setzt zwar Leitzins fest und kauft Anleihen, allerdings gibt es noch andere Einflüsse. ~\\
			$\Rightarrow$ Preis für risikoloses Instrument kann empirisch nicht als exogen gegeben angenommen werden.
		\end{proof}
\end{enumerate}

\newpage

\subsection*{Aufgabe 2}
Gehen Sie von einer Ökonomie mit den Zeitpunkten $t = 0$, $t = 1$ und $t = 2$ ($t$ in Jahren) aus. Die Zustandsübergänge und die Entwicklung des Konsums des repräsentativen Investors sind gemäß dem Baumdiagramm in der Aufgabe zu entnehmen. Der Nutzen des repräsentativen Investors zum jeweiligen Zeitpunkt t lässt sich mit Hilfe der folgenden Powernutzenfunktion mit $\gamma = 1.5$ quantifizieren:

$$ u(c_t) = \frac{c_t^{1-\gamma}}{1 - \gamma} $$

Nehmen Sie für die zeitliche Diskontierung $\beta = 0.95$ an.
\begin{enumerate}
	\item Für die Zustandsübergangswahrscheinlichkeiten gilt zunächst $p_1 = p_{21} = p_{22} = 0.5$. Bestimmen Sie die beiden fairen risikolosen Zinssätze (annualisiert) für die Zeiträume $t = 0$ bis $t = 1$ und $t = 0$ bis $t = 2$. Berechnen Sie außerdem den erwarteten Gesamtnutzen des repräsentativen Investors zum Zeitpunkt $t = 0$.
		\begin{proof}
		Es ist $u'(c_t) = c_t^{-\gamma}$ und damit (vgl. Cochrane, S. 24f)
		$$ R_{t+1}^f = \frac{1}{\mathbb{E}\left[ \beta \frac{u'(c_{t+1})}{u'(c_t)} \right]} = \frac{1}{\mathbb{E}\left[ \beta \left( \frac{c_{t}}{c_{t+1}} \right)^\gamma \right]} $$
		Damit ist
			\begin{align*}
				R_{t+1}^f & = \frac{1}{\mathbb{E}\left[ \beta \left( \frac{c_{t}}{c_{t+1}} \right)^\gamma \right]} \\
				& =\frac{1}{0.5 \cdot 0.95  \left( \frac{20}{25} \right)^{1.5} + 0.5 \cdot 0.95  \left( \frac{20}{21} \right)^{1.5} } \\
				& = 1.2798 ~ \hat{=} ~ 27.98\% \\
			\end{align*}
			\begin{align*}
				R_{t+2}^f & = \frac{1}{\mathbb{E}[m_{0,2}]} = \frac{1}{\mathbb{E}[m_{0,1} \cdot m_{1,2}]}  \\
				& = \frac{1}{\mathbb{E}\left[\beta \cdot \frac{u'(c_1)}{u'(c_0)} \cdot \beta \cdot \frac{u'(c_2)}{u'(c_1)} \right]} \\
				& = \frac{1}{\mathbb{E}\left[ \beta^2 \left( \frac{c_{t}}{c_{t+2}} \right)^\gamma \right]} \\
			& =\frac{1}{0.25 \cdot 0.95^2 \left( \left( \frac{20}{30} \right)^{1.5} +  \left( \frac{20}{26} \right)^{1.5} + \left( \frac{20}{26} \right)^{1.5} + \left( \frac{20}{22} \right)^{1.5} \right)} \\
			& = 1.6056~ \hat{=} ~60.56\%
			\end{align*}
			Damit liegt das Annulalisierte bei $r^f = \sqrt{R_{t+2}^f} = \sqrt{~60.56} = 26.61\%$. Der erwartete Nutzen ergibt sich zu
			\begin{align*}
				U_t(c_t, c_{t+1}, c_{t+2}) & = u(c_t) + \beta \mathbb{E}[ u(c_{t+1})] + \beta^2 \mathbb{E}_t \left[ u(c_{t+2}) \right] \\
						& = -2 \left( \frac{1}{\sqrt{20}} + 0.95 \left( 0.5 \cdot \frac{1}{\sqrt{25}} + 0.5 \cdot \frac{1}{\sqrt{21}} \right)\right) \\
						& \quad-2 \left( 0.95^2 \left( 0.25 \cdot \frac{1}{\sqrt{30}} + 0.25 \cdot \frac{1}{\sqrt{26}} + 0.25 \cdot \frac{1}{\sqrt{26}} + 0.25 \cdot \frac{1}{\sqrt{22}} \right)  \right) \\
						& = - 1.2001
			\end{align*} 
		\end{proof}
	\item Nehmen Sie nun an, dass für die Zustandsübergangswahrscheinlichkeiten $p_1 = 0.5$, $p_{21} = 0.7$ und $p_{22} = 0.3$ gilt. Würde der repräsentative Investor diese Situation der Situation aus Aufgabenteil a) vorziehen? Antworten Sie sowohl mit einer ökonomischen Argumentation als auch mit einer Rechnung.
		\begin{proof}
			Vergleiche den Gesamtnutzen des Investors zwischen a) und b)
				\begin{align*}
				U_t(c_t, c_{t+1}, c_{t+2}) & = u(c_t) + \beta \mathbb{E}[ u(c_{t+1})] + \beta^2 \mathbb{E}_t \left[ u(c_{t+2}) \right] \\
						& = -2 \left( \frac{1}{\sqrt{20}} + 0.95 \left( 0.5 \cdot \frac{1}{\sqrt{25}} + 0.5 \cdot \frac{1}{\sqrt{21}} \right)\right) \\
						& \quad-2 \left( 0.95^2 \left( 0.5 \cdot 0.7 \cdot \frac{1}{\sqrt{30}} + 0.5 \cdot 0.3 \cdot \frac{1}{\sqrt{26}} + 0.5 \cdot 0.3 \cdot \frac{1}{\sqrt{26}} + 0.5 \cdot 0.7 \cdot \frac{1}{\sqrt{22}} \right)  \right) \\
						& = - 1.2007 < -1.2001
			\end{align*} 		
			Nutzen in $t = 2$ ist volatiler (den Wahrscheinlichkeit bei 30 und 22 zu Lasten). Investoren bevorzugen allerdings einen konstanten Konsum
			$$ \gamma = 1.4 \Rightarrow \text{ risikoavers} $$
			$\Rightarrow$ ziehe a) vor.
		\end{proof}
	\item Welche Auswirkung hätte eine Veränderung der Zustandsübergangswahrscheinlichkeiten gemäß Aufgabenteil b) (im Vgl. zu Aufgabenteil a)) auf den risikolosen Zinssatz von t = 0 bis t = 2? Antworten Sie ohne Rechnung und gehen Sie insbesondere darauf ein, über welchen Kanal (erwarteter Konsumwachstum vs. Volatilität des Konsumwachstums) sich die veränderten Übergangswahrscheinlichkeiten auf den Zinssatz in diesem Fall auswirken.
		\begin{proof}
			Durch die Änderung der Wahrscheinlichkeiten in b) ändert sich der Erwartungswert des Konsums und damit Erwartungswert von $\Delta c_2$ nicht. Was sich geändert hat, ist die Volatilität des Konsumwachstums. Der zukünftige Zustand ist nun unsicher, wodurch man vorzüglich spart (precautionary savings). Durch diese Tatsache erhöht sich die Nachfrage und im Zuge dessen reduzieren sich die Zinsen.
		\end{proof}
	\item Können die risikoneutralen Wahrscheinlichkeiten in der gegebenen Ökonomie berechnet werden? Gehen Sie bei der Beantwortung der Frage darauf ein, wie sich die gegebene Situation im Vergleich zu den Standardsituationen, im Rahmen derer risikoneutrale Wahrscheinlichkeiten in Vorlesung und Übung bestimmt wurden, unterscheidet.
		\begin{proof}
			Rein theoretisch könnten die risikoneutralen Wahrscheinlichkeit für beide Stufen anhand der Formel berechnet werden.
			$$ \pi^*(s) = \frac{m(s)}{\mathbb{E}[m]} \cdot \pi(s) = R^f \cdot pc(s) = R^f \cdot m(s) \cdot \pi(s) $$
			wobei $R^f = R^f_{01}/R_{02,}^f$ und $m(s) = \beta \cdot \frac{u'(c(s))}{u'(c_0)}$ bzw. $m(s)= \beta ^2\cdot \frac{u'(c(s))}{u'(c_0)}$. ~\\
			\textbf{Unterschiede}:
			\begin{itemize}
				\item Es liegt ein zweistufiges Modell vor
				\item In der vorliegenden Situation liegen keinerlei Informationen über Wertpapiere vor (Preis, Auszahlungsprofil, $\dotsc$). ~\\
					Lediglich die Konsumniveaus sind bekannt. ~\\
					Risikoneutrale Wahrscheinlichkeiten im Zusammenhang mit Derivatenbewertung
			\end{itemize} 
			Bemerkung: würde man hier risikonetutrale Wahrscheilichkeiten berechnen? Nein, da hier einfach Erwartungsnutzen ausgerechnet werden kann $\Rightarrow$ Vortil risikoneutrale Wahrscheinlichkeiten $\Rightarrow$ keine Aussage über Nutzenfunktion notwendig (diese ist in Aufgabe 2 atypischerweise gegeben).
		\end{proof}
\end{enumerate}

% Index									
\renewcommand{\indexname}{Stichwortverzeichnis}
\printindex


\end{document}